\documentclass[10pt,landscape]{article}
\usepackage{amssymb,amsmath,amsthm,amsfonts}
\usepackage{multicol,multirow}
\usepackage{calc}
\usepackage{ifthen}
\usepackage[landscape]{geometry}
\usepackage[colorlinks=true,citecolor=blue,linkcolor=blue]{hyperref}

\ifthenelse{\lengthtest { \paperwidth = 11in}}
    { \geometry{top=.5in,left=.5in,right=.5in,bottom=.5in} }
	{\ifthenelse{ \lengthtest{ \paperwidth = 297mm}}
		{\geometry{top=1cm,left=1cm,right=1cm,bottom=1cm} }
		{\geometry{top=1cm,left=1cm,right=1cm,bottom=1cm} }
	}
\pagestyle{empty}
\makeatletter
\renewcommand{\section}{\@startsection{section}{1}{0mm}%
                                {-1ex plus -.5ex minus -.2ex}%
                                {0.5ex plus .2ex}%x
                                {\normalfont\large\bfseries}}
\renewcommand{\subsection}{\@startsection{subsection}{2}{0mm}%
                                {-1explus -.5ex minus -.2ex}%
                                {0.5ex plus .2ex}%
                                {\normalfont\normalsize\bfseries}}
\renewcommand{\subsubsection}{\@startsection{subsubsection}{3}{0mm}%
                                {-1ex plus -.5ex minus -.2ex}%
                                {1ex plus .2ex}%
                                {\normalfont\small\bfseries}}
\makeatother
\setcounter{secnumdepth}{0}
\setlength{\parindent}{0pt}
\setlength{\parskip}{0pt plus 0.5ex}
% -----------------------------------------------------------------------

\title{Windenergienutzung I Cheat Sheet}

\begin{document}

\raggedright
\footnotesize

\begin{center}
     \Large{\textbf{Windenergienutzung I - Cheat Sheet}} \\
\end{center}
\begin{multicols}{3}
\setlength{\premulticols}{1pt}
\setlength{\postmulticols}{1pt}
\setlength{\multicolsep}{1pt}
\setlength{\columnsep}{2pt}

\section{Statistik}
\subsection{Basics}
Mittelwert $\bar{x}=\sum^n_{i=1}x_i$\\
Varianz $\sigma^2 = \sum^n_{i=1}(x_i-\bar{x})^2$\\
Kovarianz \\
${\displaystyle r_{xy}:={\frac {\sum _{i=1}^{n}(x_{i}-{\overline {x}})(y_{i}-{\overline {y}})}{\sqrt {\sum _{i=1}^{n}(x_{i}-{\overline {x}})^{2}\cdot \sum _{i=1}^{n}(y_{i}-{\overline {y}})^{2}}}}} ,\quad r_{xy}\in[-1,1]$\\
Lineare Korrelation zweier Messreihen wenn $|r_{xy}|\approx 1$
\subsection{Lineare Regression}
Geradengleichung $y=b_1\cdot x +b_0$\\
Koeffizienten ${\displaystyle {\begin{aligned}b_{1}&={\frac {\sum \nolimits _{i=1}^{n}(x_{i}-{\overline {x}})(y_{i}-{\overline {y}})}{\sum \nolimits _{i=1}^{n}(x_{i}-{\overline {x}})^{2}}}\\b_{0}&={\overline {y}}-b_{1}{\overline {x}}\end{aligned}}}$\\


\section{Berechnung von Wind}
\begin{tabular}{lll}
	Rauhigkeitsh\"ohe & $z_0$ & m \\
	Messh\"ohe & $z_1$ & m \\
	Nabenh\"ohe & $z_2$ & m \\
	Universelle Gaskonstante & $R$ & $\text{m}^2/\text{s}^2$K \\
	Radius & $r$  & m\\
	Dichte & $\rho$  & kg/$\text{m}^3$\\
	Rauhigkeitsexponent & $\alpha$ & - \\
	Widerstandsbeiwert & $c_W$  & - \\
	Auftriebsbeiwert & $c_A$  & - \\
	Windrichtung & $\gamma$  & $\circ$\\
	Jahresmittel der Windgeschwindigkeit & $v_m$  & m/s\\
	Mittlere Windgeschwindigkeit (10 min) & $\bar{v}$ & m/s\\
\end{tabular}
Logarithmische Geschwindigkeitsverteilung $v_2 = v_1 \cdot \frac{\text{ln}(z_2/z_0)}{\text{ln}(z_1/z_0)}$\\
Rauhigkeitsexponent $z_0 \approx 15.25\text{m} \cdot \text{exp}(-\frac{1}{\alpha})$\\
Powerlaw Geschwindigkeitsverteilung $v_2 = v_1 \cdot (\frac{z_2}{z_1})^\alpha$\\
Ideale Gasgleichung $\rho = \frac{p}{R\cdot T}$\\
Windleistung $P_{\text{Wind}} = \frac{\rho}{2} v_1^3 \pi r^2$\\
Leistungsbeiwert $c_P = \frac{P_{\text{Rotor}}}{P_{\text{Wind}}}$\\
Maximaler Leistungsbeiwert $c_{P,max}=\frac{16}{27}\cdot \text{cos}^3\gamma$
Turbulenzintensit\"at $I_T = \frac{\sigma_v}{\bar{v}}$\\

\subsection{Weibullverteilung}
\begin{tabular}{lll}
	Skalierungsparameter & $A$ & m/s \\
	Formfaktor & $k$ & - \\
\end{tabular}
Verteilungsdichtefunktion $f(v)=\frac{k}{A}(\frac{v}{A})^{k-1}\cdot \text{exp}(-(\frac{v}{A})^k)$\\
Kumulative Verteilung $F(v)=1-\text{exp}(-(\frac{v}{A})^k)$ \\
Jahresmittel Windgeschwindigkeit $v_m \approx A \cdot (0.568+\frac{0.434}{k})^\frac{1}{k}$ \\

%Gammafunktion $\Gamma(x) = \int_0^{\infty}\text{exp}(-t)t^{x-1} dt$
% 1. + 2. Moment Wind II Folie 14

Annahme bei Umrechnung $k_2 = k_1$ \\
Logarithmisches Windprofil $A_2 = A_1 \cdot \frac{\text{ln}(z_2/z_0)}{\text{ln}(z_1/z_0)}$\\

\subsection{Gumbelverteilung}
F\"ur sehr hohe Windgeschwindigkeiten.\\
Euler Konstante $\gamma=0.5772$ \\
Kumulative Verteilung $F_{\text{Gumbel}}(v) = \text{exp}\left[-\text{exp}(-\gamma+\frac{\pi}{\sqrt{6}}\frac{v-m_{\text{extreme}}}{\sigma_{\text{extreme}}})\right]$\\
Gumbelparameter aus Weibull ${\displaystyle {\begin{aligned}m_{\text{extreme}}=\frac{\gamma-k\cdot\text{ln}(A)}{k}\\\sigma_{\text{extreme}}=\frac{\pi}{k\cdot\sqrt{6}}\end{aligned}}}$\\

\subsection{Brutto Jahresenergieertrag}
\begin{tabular}{lll}
	Zeitraum & $T$ & s \\
	Leistung & $P$ & W \\
\end{tabular}\\
Jahresenergieertrag $AEP = 8760\text{h} \cdot \sum^n_{i=1}\left[(F(v_i)-F(v_{i-1}))(\frac{P_{i-1}+P_i}{2})\right]$\\
Energieertrag (Weibull) $E_i=f(v_i) \cdot P_i \cdot T , \quad E_{\text{ges}}=\sum E_i$

\subsection{Netto Jahresenergieertrag}
Netto AEP $\approx$ Brutto AEP - 10$\%$ durch Verschmutzung, Wartung, Netzverluste und Windparkeffekte.

\subsection{Auslastung}
Der Anzahl der j\"ahrlichen Vollaststunden $Q_1$ und der Auslastungsfaktor $Q_2$ sind ein Ma$\ss$ f\"ur die G\"ute des Standortes und die Anpassung der Anlage.\\
Volllaststunden $Q_1 = \frac{\text{Tats\"achlicher Jahresertrag [kWh]}}{\text{Nennleistung [kW]}}$\\
Auslastungsfaktor $Q_2 = \frac{Q_1}{8760\text{h}}$\\
Der Auslegungsfaktor sollte idealerweise zwischen 20-70\% liegen, normalerweise liegt er um die 25-30\%.


\section{Aerodynamik}
\subsection{Energie und Leistung}
\begin{tabular}{lll}
	Fl\"ache & $F$ & $\text{m}^2$ \\
	Geschwindigkeit vor Rotor & $v_1$ & m/s \\
	Geschwindigkeit am Rotor & $v_2$ & m/s \\
	Geschwindigkeit nach Rotor & $v_3$ & m/s \\
\end{tabular}\\
Kinetische Energie $E=\frac{1}{2}\cdot m \cdot v_1^2$\\
Massendurchsatz $\dot{m}=\rho\cdot F \cdot v$\\
Windleistung $P_{\text{Wind}}=\dot{E}=\frac{1}{2}\cdot\rho\cdot F \cdot v_1^3$\\
Kontinuit\"atsbedingung $\dot{m} = const$\\
Rotorschub $S=\dot{m}\cdot(v_1-v_3)$\\
Windgeschwindigkeit am Rotor $v_2 = (v_1+v_3)/2$

\subsection{Maximaler Wirkungsgrad nach Betz}
Leistungsbeiwert $c_P = \frac{1}{2}\cdot(1+\frac{v_3}{v_1})\cdot(1-(\frac{v_3}{v_1})^2)$\\
Rotorleistung $P_{\text{Rotor}} = P_{Wind} \cdot c_P$\\
Optimale Abbremsung wenn $v_3 = v_1/3$, damit folgt $v_2 = 2/3\cdot v_1$
am Rotor. Daraus ergibt sich der maximale Leistungsbeiwert zu $c_{P,\text{Betz}}=\frac{16}{27}=0.593$\\

\subsection{Actuator Disk Theorie}
Abrupter statischer Druckabfall am Rotor durch Rotorschub.
\begin{tabular}{lll}
	Axialer Induktionsfaktor  & $a$ & - \\
\end{tabular}\\
Induzierte Geschwindigkeit $v_i=v_1-v_2=a \cdot v_1$\\
Schubbeiwert $c_S = \frac{\text{Schub}}{\text{Staudruck} \cdot \text{Fl\"ache}} = \frac{S}{\frac{\rho}{2}\cdot v_1^2\cdot F} = 4a\cdot(1-a)$
Betz-Optimum bei $a = 1/3 ,\quad c_S = 8/9$\\
Rotorleistung $P_{\text{Rotor}} = \frac{\rho}{2} \cdot F \cdot v_1^3 \cdot (4a\cdot (1-a)^2)$\\
Leistungsbeiwert $c_P = 4a \cdot (1-a)^2$\\

\subsection{Dimensionslose Kenngr\"o$\ss$en}
\begin{tabular}{lll}
	Winkelgeschwindigkeit  & $\Omega$ & rad/s \\
	Rotordrehmoment & M & Nm \\ 
	Kinematische Viskosit\"at & $\nu$ & $m^2/s$\\
	Profiltiefe & t & Nm \\ 
\end{tabular}\\
Schnelllaufzahl $\lambda=\frac{r\cdot\Omega}{v_1},\quad \lambda\approx1 \text{ Kraft}, \quad \lambda>5 \text{ Leistung}$
Momentenbeiwert $c_M = \frac{M}{\frac{\rho}{2}v_1^2\pi r^3} = \frac{c_P}{\lambda}$\\
Reynoldszahl $Re = \frac{\text{Tr\"agheitskraft}}{\text{Z\"ahigkeitskraft}} = \frac{v \cdot t}{\nu}$

\subsection{Auftrieb und Widerstand}
\begin{tabular}{lll}
	Gravitationskonstante  & $g$ & m/$s^2$ \\
	H\"ohe  & $z$ & m \\
	Fl\"ugelfl\"ache & S & $m^2$ \\
\end{tabular}\\
Bernoulli Gleichung $p + \rho gz + \frac{\rho}{2}v^2 = const.$\\
Auftrieb $A = \frac{\rho}{2} c_A v^2 S $\\
Widerstand $W = \frac{\rho}{2} c_W v^2 S $\\
Gleitzahl $\epsilon = c_W/c_A$

\subsection{Rotorauslegung nach Betz}
Vorgabe der Entwurfsschnelllaufzahl $\lambda_A$, Auftriebsbeiwertes $c_A$, Entwurfsanstellwinkels $\alpha_A$ und der Anzahl der Bl\"atter $z$. Ergebnis ist dann der Profiltiefenverlauf $t(r)$ und die Profilwindung $\alpha(r)$.\\
\begin{tabular}{lll}
	Dimensionsloser Radius  & $r/R$ & - \\
	Rotorradius & $R$ & m \\
\end{tabular}\\
Profiltiefenverteilung $t(r) = 2\pi R \frac{1}{z} \cdot \frac{8}{9\cdot c_a} \cdot \frac{1}{\lambda_A \cdot \sqrt{\lambda_A^2 \cdot (\frac{r}{R})^2 + \frac{4}{9}}}$\\
Profilwindung $\alpha(r) = \text{arctan}(\frac{2}{3\cdot\lambda_A\cdot r/R})-\alpha_A$\\

\subsection{Rotorauslegung nach Schmitz}
Drallverluste aus Diagram $\propto \frac{1}{\lambda_A^2}$\\
Profilwiderstandsverluste $\eta_{\text{Profil}} = 1-\frac{3}{2}\frac{r}{R}\frac{\lambda_A}{r/R}$\\
%$P_{\text{real}}=c_{P,\text{Betz}}\frac{\rho}{2}v_1^3\pi R^2(1-\frac{\lambda_A}{1/\epsilon})$\\
Tipverluste $\eta_{\text{Tip}}\approx1-\frac{1,84}{z\lambda_A}\quad\text{f\"ur }\lambda_A>2$ \\
Profiltiefenverteilung $t(r) = \frac{1}{z}\frac{16\pi}{c_a}\cdot r \cdot \text{sin}^2(\frac{\alpha_1}{3})$\\
mit $\alpha_1=\text{arctan}(\frac{R}{\lambda_A\cdot r})$\\
Unterschiede der beiden Verfahren vor allem im Blattwurzelbereich ($<1\%$ des Rotorradius). \\
Bestimmung des $c_{P,\text{real}}$ aus Diagramm, dann\\
$c_{P,\text{max,real}} = c_{P,\text{real}} \cdot \eta_{\text{Profil}} \cdot \eta_{\text{Tip}} < c_{P,\text{Betz}}$

\section{Strukturdynamik}
\subsection{Einteilung von Belastungen}
\begin{tabular}{lll}
	Station\"ar  & Eigengewicht, Fliehkraft &  \\
	Transient & Bremskr\"afte &  \\
	Periodisch & Massenunwucht &  \\
	Stochastisch & Windturbulenz, B\"oen &  \\
	Kraft & F & N \\
	Hebelarm & L & m\\
	Zeit & t & s\\
\end{tabular}\\
Moment an der Gondel $M=F\cdot L\cdot\text{sin}(\Omega\cdot t)$\\
$\rightarrow$ Periodische Anregung der Gondel durch $\Omega$\\
Periodisches Gondelmoment bei Ein- und Zweiblattrotoren. Bei Drei- und Vierblattrotoren l\"oscht sich die Periodizit\"at aus, sodass ein konstantes Gondelmoment vorliegt, solange konstante Windverh\"altnisse vorliegen.

\subsection{Rotational Sampling}
Jede B\"oe wird mehrmals durch die Rotorbl\"atter durchschnitten. Aus Sicht des Blattes wird die Belastung durch die B\"oe so wiederholt abgetastet. Die Anregung der B\"oe wird somit mit der Rotationsfrequenz des Blattes \"uberlagert.



\section{Konstruktiver Aufbau}
Auslegungslebensdauer um die 20 Jahre.\\
Derzeit praktisch nur Dreiblattrotor.\\
\subsection{Nabenkonzepte}
\begin{itemize}
	\item Starr mit/ohne Blattverstellung
	\item Gelenkig, schlagend oder pendelnd
	\item Growian-Nabe kombiniert Pendeln und Pitchverstellung
\end{itemize}



\section{Elektrische Systeme}
\subsection{Elektrische Komponenten}
\begin{itemize}
	\item Generator
	\item Rotor PM + Stator
	\item Encoder
	\item Frequenzumrichter
	\item Transformatoren
\end{itemize}

\subsection{Generatoren}
Unterscheidung in Synchron- (GS) und Asynchron-Generatoren (ASG).
\subsubsection{Synchron-Generatoren}
\begin{itemize}
	\item Der L\"aufer (Elektro- oder Permanentmagnet) induziert eine mit der Drehzahl umlaufende Spannung in den St\"ander.
	\item Der L\"aufer eilt mit dem lastabh\"angigen Polradwinkel dem mit der Netzfrequenz umlaufenden St\"anderfeld vor.
	\item \textbf{Drehzahl ist konstant und gleich der Synchrondrehzahl.}
\end{itemize}

\subsubsection{Asynchron-Generatoren}
\begin{itemize}
	\item Sehr robuster und preisg\"unstiger Generator, durch geringf\"ugige Drehzahlflexibilit\"at gut f\"ur Windenergieanlagen geeignet
	(sonst geringe Verwendung)
	\item Der mit dem Netz verbundene St\"ander induziert eine Spannung in den L\"aufer, wenn dieser sich \"ubersynchron, d.h. schneller als das umlaufende
	St\"anderfeld, dreht. Das Rotormagnetfeld induziert wiederum einen Strom in
	den Stator.
	\item Die Stromst\"arke h\"angt von der Differenzdrehzahl, dem Schlupf, ab.
	\item Nennschlupf ca. 1\%, kann durch gr\"osseren L\"auferwiderstandes erh\"oht
	werden.
\end{itemize}
Schlupf $s = \frac{n_{syn}-n}{n_{syn}} $ 

\subsection{Transformatoren}
Die Funktion des Trafos ist die Ausgangsspannung des Generators (typisch 690 V) auf Mittelspannung (10-36 KV) zu heben.\\
\vspace{7pt}
Zwei Trafo Typen - Trocken und mit \"Ol gefüllt (synthetisches \"Ol, Silikon\"ol, Ester); Unterschiede in Wartungsaufwand, Flammbarkeit, K\"uhleigenschaften, Isolationseigenschaften, Kompaktbauweise und Kosten.\\
\vspace{7pt}
\textbf{Giessharztransformatoren} zeichnen sich aus durch feuerbeständigkeit, hohe Isolierung und Kurzschlussfähigkeit, Unempfindlichkeit in tropischen Umgebungen.\\
\vspace{7pt}
\textbf{Fl\"ussigkeitsgef\"ullte Transformatoren} zeichnen sich aus durch tiefere Leerlauf und Wicklungsverluste, geringere Abmessung und Gewicht, günstig und recyclebar. Sie sind vor Allem an Standorten ohne Brandschutzauflagen verwendbar.

\subsection{Schaltanlagen}
Mittelspannung Schaltanlagen k\"onnen in drei Typen aufgeteilt werden: luftisoliert, gasisoliert (SF6, Schwefelhexafluorid) und feststoffisoliert.

\subsection{Schaltungskonzepte}
\subsubsection{1. Traditionell D\"anisch}
\begin{itemize}
	\item Direkte Netzkoppelung eines Asynchrongenerators (Das Netz erzwingt konstante Drehzahl.)
	\item Kombination mit polumschaltbarem Generator (2 Drehzahlen)
	\item Stall- oder Aktiv-Stallregelung
	\item statische Blindleistungskompensation
\end{itemize}

\subsubsection{2. Asynchrongenerator mit Schlupfregelung}
\begin{itemize}
	\item Direkte Koppelung eines Asynchrongenerators an das Netz
	\item Drehzahlvariabilit\"at von 0\% bis +10\% durch Ver\"anderung
	des Widerstands im Rotorkreis („Opti-Slip“).
	\item Pitchregelung
	\item statische Blindleistungskompensation
\end{itemize}

\subsubsection{3. Doppelt-gespeister Asynchrongenerator}
\begin{itemize}
	\item Direkte Koppelung des Stators des Asynchrongenerators an das Netz -> im Stator Netzfrequenz
	\item  Umrichter zwischen dem Rotor des Asynchrongenerators und dem Netz
	\item  Volle Drehzahlvariabilit\"at von ca. -33\% bis +33\% $n_{syn}$ durch ver\"anderliche Frequenz
	im Rotor durch 4-Quadranten-Umrichter nur f\"ur Leistung aus dem Rotor (bis 1/3 des
	Gesamtleistung) erm\"oglicht Umkehr der Energieflussrichtung
	\item  Pitchregelung
\end{itemize}

\subsubsection{4. Synchrongenerator mit variabler Drehzahl und Vollumrichter}
\begin{itemize}
	\item Drehmoment wird \"uber die Erregung geregelt
	\item  Vollumrichter erm\"oglicht Einspeisung mit konstanter Netzfrequenz
	\item  Pitchregelung,
\end{itemize}


\section{Regelungstechnik}

\subsection{Hierarchie}
\begin{enumerate}
	\item \textbf{Sicherheitssystem} f\"ur Not-Aus Funktion im "Fail Safe" Mode
	\item \textbf{Regelung} zur Leistungsbegrenzung und Optimierung der Teillast
	\item \textbf{Betriebsf\"uhrung} f\"ur Start, Stopp, Hilfsantriebe und Windnachf\"uhrung
	\item \textbf{Fern\"uberwachung (SCADA)} f\"ur Betriebsdatenerfassung und Remote Control
\end{enumerate}


\end{multicols}

\section{Editorial}
Created by Christian Molli\`ere.\\
Last updated \today.\\
Feel free to share and edit!

\end{document}